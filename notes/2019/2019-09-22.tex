\chapter{2019-Sep-22}
\marginnote{\Large{\textbf{Sunday}}}
\marginnote{\Large{\textbf{Week: 38}}}

\section{Cengel-Boles}

\subsection{Chapter 1 Introduction and Basic Concepts}

\begin{ennumerate}
\item	one does not need to know the behavior of the gas particles to determine the pressure in the container. It would be sufficient to attach a pressure gage to the container. This macroscopic approach to the study of thermodynamics that does not require a knowledge of the behavior of individual particles is called classical thermodynamics. It provides a direct and easy way to the solution of engineering problems. A more elaborate approach, based on the average behavior of large groups of individual particles, is called statistical thermodynamics.

\item	closed system or control mass: No mass transfer

isolated system: No mass or energy transfer.

open system or control volume: Both mass and energy can cross the boundary of a control volume.

\item	Any characteristic of a system is called a property. Properties are considered to be either intensive or extensive. Intensive properties are those that are independent of the mass of a system, such as temperature, pressure, and density. Extensive properties are those whose values depend on the size-or extent-of the system. Total mass, total volume, and total momentum are some examples of extensive properties. Extensive properties per unit mass are called specific properties.

\item	Matter is made up of atoms that are widely spaced in the gas phase. Yet it is very convenient to disregard the atomic nature of a substance and view it as a continuous, homogeneous matter with no holes, that is, a continuum. The continuum idealization allows us to treat properties as point functions
and to assume the properties vary continually in space with no jump discontinuities. This idealization is valid as long as the size of the system we deal with is large relative to the space between the molecules. This is the case in practically all problems, except some specialized ones. The continuum
idealization is implicit in many statements we make, such as “the density of water in a glass is the same at any point.”

Despite the relatively large gaps between molecules, a gas can usually be treated as a continuum because of the very large number of molecules even in an extremely small volume.

\end{ennumerate}

\subsection{Fun Facts}
A typical match yields about one Btu (or one kJ) of energy if completely burned.

Note that kW or kJ/s is a unit of power, whereas kWh is a unit of energy
